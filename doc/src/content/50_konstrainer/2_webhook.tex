\section{Webhooks}

The report can be used to diagnose the cluster, and the webhooks can be used to enforce rules by rejecting or altering modifications that result in an undesired state.

The main keyword for creating a webhook rule is the `webhook'. A webhook requires a name, which must be unique on the cluster level and must be in kebab-case.

A webhook definition has two main parts: the configuration and the behavior block.

\subsection{Webhook configuration}

The configuration defines which events the webhook listens to. For example, we can create a webhook rule, that listens to \emph{Deployment} and \emph{DaemonSet} creations and updates, like demonstrated in the \ref{code:wh_conf} example.

\begin{lstlisting}[caption={Webhook configuration},language=Kotlin,label=code:wh_conf]
webhook("webhook-unique-name") {
  operations(CREATE, UPDATE)
  apiGroups(APPS)
  apiVersions(ANY)
  resources(DEPLOYMENTS, DAEMONSETS)
  failurePolicy(FAIL)
  namespaceSelector {
    matchLabels = mapOf(
      "managed" to "true"
    )
  }
  logRequest = true
  logResponse = true

  behavior {
    /* Behavior goes here */
  }
}
\end{lstlisting}

Note that the webhooks and the report have slightly different syntax and design. The main reason for this is that I designed the webhook language feature earlier than the report. When I was designing the webhooks, I was still figuring out how to create a Kotlin DSL, and I also had to experiment with the concept of dynamic admission control in Kubernetes. This resulted in a less mature design. By the time I got to the reporting functionality, I had more experience, allowing me to perfect the design. The current design of the webhooks feature is not as polished as the report, but it is useable even in real-life scenarios. Therefore, I leave the redesigning of the webhooks as a future improvement.

I more or less created a mapping for the Kubernetes \emph{MutatingWebhookConfiguration}
\footnote{\url{kubernetes.io/docs/reference/access-authn-authz/extensible-admission-controllers}}
object, but I tried to simplify it. The most significant simplification, which makes it much easier to use, is that I totally hid the SSL and implementation-related parts, like the `caBundle' and `clientConfig' fields. These are simplifications for the user, and they do not reduce the functionality; they just make it easier to use the webhooks. However, I also made a simplification that slightly reduces functionality. The \emph{MutatingWebhookConfiguration} object has the rules' field, which is a list of \emph{RuleWithOperations} objects. I do not allow it to be a list, I only allow it to be a single element. So the `operations', `apiGroups', `apiVersion', `resources', `scope' fields could be set multiple times in the rules list, but I just did not implement it in such a way. This could be a future improvement.

In the \ref{code:wh_conf} example, we can see the syntax for setting these fields. I decided to use the `fieldName(param, param2, ...)' syntax. These are essentially functions with vararg parameters\footnote{\url{kotlinlang.org/docs/functions.html\#variable-number-of-arguments-varargs}}, allowing you to list your parameters separated by commas, and they will be converted to an array. I chose this approach because these fields are mostly lists. Even though Kotlin has a nice way of creating lists with the \lstinline|listOf| function, I did not want to use it. I thought that using the setter function syntax would make scripts look more like a DSL and less like regular Kotlin code. However, looking back, I am not sure if this approach is better or not.

I also created keyword constants for the most commonly used string literals in this context, such as `CREATE', `DELETE', `ANY' or `DEPLOYMENTS', however string literals can also be used.

The \ref{tree:str_literals} figure shows the complete list of constants, their values, and the contexts in which they can be used.

\subfile{str_literals.tex}

Let's talk about the `namespaceSelector'. I will only write about it on the DSL level, to understand what it actually does, see the official documentation\footnote{\url{kubernetes.io/docs/reference/access-authn-authz/extensible-admission-controllers/\#matching-requests-namespaceselector}}. 
This keyword opens a new block with the LabelSelector class from the fabric8 package as the receiver. The created LabelSelector will be passed to the MutatingWebhookConfiguration as is. This gives full control, but also requires the user to fully understand the topic. This is also an aspect that could be improved in the future.

The \ref{code:wh_conf} example also shows two fields: `logRequest' and the `logResponse'. These are tools for developing and debugging your webhooks. If they are set to true, the incoming webhook request or the given response is logged in a JSON format by the agent. These logs can be used to troubleshoot your webhooks, however I recommend turning them off in production.

\subsection{Webhook configuration bundle}

Webhooks require a lot of configuration. To eliminate repetition, I created a language feature, which allows for reusable configurations. This can be done via the `webhookConfigBundle' keyword. See the \ref{code:wcb} code example on how to use it.

\begin{lstlisting}[caption={Webhook configuration bundle},language=Kotlin,label=code:wcb]
val conf = webhookConfigBundle {
  operations(CREATE, UPDATE)
  apiGroups(APPS)
  resources(ANY)
}
val myServer = server("my-server") {
  webhook("wh-example", conf) {
    resources(DAEMONSETS)
  }
}
\end{lstlisting}

The `webhookConfigBundle' is a top-level keyword, similar to the server.' Within its block, you have access to all language features provided by the `webhook' block, except for the `behavior' block. To use the configuration bundle inside a webhook, pass it as a second parameter. When a configuration bundle is provided in a webhook, the default values are set based on the bundle. These default values can be overridden within the webhook block, as shown in the \ref{code:wcb} example, where the `resources' field is overridden.

\subsection{Behavior block}

Everything so far about the webhook block was processed at startup time, however the behavior block runs when the webhook is triggered. The implementation of it is detailed in TODO chapter. 

To understand the structure of the behavior block, let's quickly review how an admission webhook works. This is detailed in the \ref{sec:admcont} section of the \nameref{chap:prerequisites} chapter, but basically when a request arrives to the webhook, the webhook processes the request, makes a decision, than assembles a response.

TODO

\subsubsection{Processing request}

For processing requests, I implemented several tools. The request arrives in the form of an \emph{AdmissionRequest}\footnote{\url{kubernetes.io/docs/reference/config-api/apiserver-admission.v1/}} object. The only field required for the custom logic we want to define in the \emph{AdmissionRequest} object is the `request' field. The behavior block provides access to this field using the `request' keyword, represented as a \emph{kotlinx.serialization.json.JsonObject} object. Since processing it with regular Kotlin tools is cumbersome, I have provided two main approaches: unmarshalling the request into Java objects from the fabric8 library, and a jq\footnote{jq is a lightweight and flexible command-line JSON processor, \url{jqlang.github.io/jq/}}-like JSON processor language feature.

Now, let's discuss the JSON processor. It offers a way to null-safely read data from \emph{JsonElement}s and convert the end result to a string, number, or boolean.

The keyword for reading data is `jqx'. It functions as an infix operator, with a \emph{JsonElement} as the left-hand side argument and a selector string as the right-hand side (RHS) argument. The selector string specifies which field to retrieve from the \emph{JsonElement}. The format of the selector string uses slash-separated words, and these words can even be numbers to select from a list. Surrounding slashes do not matter. The \ref{code:jqx} example illustrates the usage of this command.

\begin{lstlisting}[caption={Usage of `jqx'},language=Kotlin,label=code:jqx]
// Pseudo-Kotlin JSON definition
val myJson = {
  "lorem": [
    {
      "ipsum": "dolor"
    },
    {
      "sit": "amet"
    }
  ]
}

val result: JsonElement = myJson jqx "/lorem/0/ipsum"
// The value of result is a JSON primitive: "dolor"
\end{lstlisting}

The `jqx' operator returns a \emph{JsonElement}. In some cases, might we only want to check if something is null or empty. Two special \emph{JsonElement} objects facilitate this: \emph{JsonNull} from the \emph{kotlinx.serialization.json} library and \emph{JsonEmpty}, defined in my DSL library. The `==' operator can be used to compare the return value of the jqx operator with these two special JSON primitives.

While the cases mentioned above are special, most of the time, parsing the \emph{JsonElement} is necessary. The `parseAs' infix keyword can handle this. Its left-hand side argument is a \emph{JsonElement}, and the right-hand side must be one of these special tokens: `string', `int', `double', or `bool'. Depending on the token passed as the right-hand side argument, the return value is either null if it couldn't be parsed, or the value of the \emph{JsonElement} parsed as the type represented by the token. The \ref{code:jqx2} snippet demonstrates how to use the `parseAs' operator alongside the `jqx' operator.

\begin{lstlisting}[caption={Usage of `parseAs'},language=Kotlin,label=code:jqx2]
// Pseudo-Kotlin JSON definition
val myJson = {
  "color": "red",
  "count": 10,
  "isRipe": true
}

val color: String? = myJson jqx "color" parseAs string // String: "red"
val rhl: Int? = myJson jqx "count" parseAs int // Integer: 10
val isRipe: Boolean? = myJson jqx "isRipe" parseAs bool // Boolean: true
val name: String? = myJson jqx "name" parseAs string // null
\end{lstlisting}

The second approach for request processing involves unmarshalling. Within the `behavior' block, we have access to the `currentObject', `oldObject', `podSpec', and `unmarshal' keywords. Due to the structure of the \emph{AdmissionRequest} object, its `object' field is frequently read, as it stores the Kubernetes resource the webhook request concerns. In cases of update and delete operations, the `oldObject' field is also used. To avoid conflicts with the reserved keyword `object' in Kotlin, the keywords `currentObject' and `oldObject' provide access to these fields.

The `currentObject' and `oldObject' properties automatically unmarshal the following resources: \emph{Deployment}, \emph{Service}, \emph{Pod}, \emph{Secret}, \emph{DaemonSet}, \emph{StatefulSet}, \emph{Job}, \emph{CronJob}, \emph{Namespace}. While the type of these properties is \emph{HasMetadata?}, they can be cast to the previously listed types.

The `podSpec' serves as a shortcut for the `spec.template.spec' fields of resources. Writing webhooks for \emph{Deployments} and \emph{StatefulSets} are common, and they all have that field, which is the spec field of the Pods they manage. It is common use-case to write rules on the managed pods of operator resources, that's why I created this shortcut. The type of `podSpec' is \emph{PodSpec?}, and it works with any resources that have the `spec.template.spec' field.

`currentObject', `oldObject' and `podSpec' are lazy properties. This means they are only evaluated if you use them, preventing errors if the referenced field does not exist and avoiding unnecessary processing time if unused. On the other hand, `unmarshal' is a function that facilitates the deserialization of any part of the `request' property.

In different scenarios, different approaches might be the best. The \ref{code:unmarshal} code snippet demonstrates all the approaches to reading the podSpec for a \emph{Deployment}.

\begin{lstlisting}[caption={Unmarshalling},language=Kotlin,label=code:unmarshal]
// request.object is a Deployment
behavior {
  val ps1: PodSpec = unmarshal(request jqx "/object/spec/template/spec")!!
  val ps2 = podSpec!!
  val ps3 = (currentObject as Deployment).spec.template.spec
}
\end{lstlisting}

All the unmarshalling methods return a nullable type. The reason for this is exception handling. Instead of throwing an exception, they return null when they can not unmarshal the object.

\subsubsection{Making a response}

To make a response, we must first make a decision about the request. The decision can be one of these things:

\begin{itemize}
    \item Allow
    \item Allow, but raise a warning
    \item Allow, but apply a patch
    \item Allow, but apply a patch and raise a warning
    \item Reject
\end{itemize}

For users, making decisions about the request and assembling a response is basically the same step, as the fields of the \emph{AdmissionResponse} object are mapped to DSL keywords. The framework handles the actual assembly of the \emph{AdmissionResponse} object. 

For the best understanding I recommend reading the official documentation~\cite{AdmissionResponseDocs} on the topic, but I will explain the most important things along the way. The \emph{AdmissionResponse} has two required fields: `uid' and `allowed'. The `uid' field is automatically filled by the framework. The `allowed' field indicates whether the admission request was permitted or not. In the DSL, there's a keyword with the same name that opens a block. Within this block, you must provide a boolean expression, and the result of this expression becomes the value of the `allowed' field. In the DSL, using the `allowed' block is optional; if omitted, it defaults to \emph{true}.

In the official documentation, the next field is the `status' field, which can be used to provide an explanation for why the request was denied. However, this field is ignored if the value of the `allowed' field is `true'. In the DSL, the `status' keyword opens a block. Inside its block, two fields can be set: `message', `code'. The `code' field allows setting an HTTP status code, while the `message' field allows providing a detailed explanation. Both have a default value; for the message, it is `Denied by Konstrainer webhook', and for the code it is `403'. While this approach offers full control, there's potential for simplification in future improvements. The word 'status' may not immediately convey its role in explaining the rejection reason. Introducing a single settable property, such as `rejectionReason,' might enhance intuitiveness.

The \ref{code:allowed} code snippet provides an example of how to use the two previously introduced keywords.

\begin{lstlisting}[caption={Usage of `allowed' and `status'},language=Kotlin,label=code:allowed]
behavior {
  allowed {
    podSpec!!.affinity != null
  }
  status {
    message = "All pods must have an affinity!"
    status = 400
  }
}
\end{lstlisting}

The next optional part of the response is the patch. It is two fields in the \emph{AdmissionResponse} k8s object, but in the DSL it is encapsulated in a single keyword, as the patchType field is either empty or has the value `JSONPatch', so it is handled automatically by the framework. 

JSON Patch is a standard used to describe small changes to a JSON document~\cite{JSONPatch}. A JSON Patch document is a JSON document containing a list of patch operations. The patch operations supported by the standard are “add”, “remove”, “replace”, “move”, “copy” and “test”.

The \ref{code:json_patch_1} code snippet shows an example JSON Patch. A patch object always has an `op' field indicating the operation. In the DSL, I created function keywords for each supported operations. A patch object also has one or two other parameters, representing the arguments of the operation. In the DSL, these are represented as input parameters for the operation functions. The \ref{code:json_patch_2} code snippet demonstrates how to implement the same patch as the \ref{code:json_patch_1} example patch.

\begin{lstlisting}[caption={JSON Patch example},language=JSON,label=code:json_patch_1]
[
  { "op": "replace", "path": "/lorem", "value": "ipsum" },
  { "op": "add", "path": "/dolor", "value": ["sit", "amet"] },
]
\end{lstlisting}

\begin{lstlisting}[caption={JSON Patch using DSL},language=Kotlin,label=code:json_patch_2]
patch {
  replace("/lorem", "ipsum")
  add("/dolor", buildJsonArray {
    add("sit")
    add("amet")
  })
}
\end{lstlisting}

The `add' and `replace' operations accept \emph{String}, \emph{Number} and \emph{JsonElement} as the second argument. For the rest of the operations it does not make sense to support anything else than \emph{String}s.

The framework automatically assembles the standard compliant JSON Patch based on the instructions in the DSL script. The resulting patch is also added as an annotation to the patched resources in a base64 encoded format, so it is clear whether a resource got patched or not. To read the patch applied to a resource run the following command: `kubectl get <resurce> -o yaml | yq '.metadata.annotations.ksr-patch' | base64 --decode'

The last feature in the response is the warning. Use the `warning' keyword to open a new block where you can list your warnings. Within the scope of the `warning' block use the `warning' keyword to raise a warning. The `warning' keyword takes only one argument: the message of the warning. To make it conditional, use the build-in language features of Kotlin, such as the `if' statement. The \ref{code:warning} code snippet provides an example of how to use warnings.

\begin{lstlisting}[caption={JSON Patch using DSL},language=Kotlin,label=code:warning]
behavior {
  val condition = /*Custom boolean expression*/
  warnings {
    if (condition) warning("My warning message")
    if (/*Custom condition*/) warning("My second warning message")
  }
}
\end{lstlisting}

As a future improvement I plan to change the warning language feature to function more like the `tag' keyword from the report, so that the condition is a second lambda parameter and not an `if' statement.

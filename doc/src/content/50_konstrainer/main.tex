\setlength{\parindent}{0pt}
\setlength{\parskip}{0.6em}

%----------------------------------------------------------------------------
\chapter[Policy language]{Policy language: Konstrainer DSL}
\label{chap:konst_dsl}
%----------------------------------------------------------------------------

This chapter will introduce my domain specific language created for enforcing Kubernetes constraints. Every keyword and language feature will be discussed with examples. In this chapter I'll explain how to use the language and if needed, explain why to use it in that way, but I will not go deep into the implementation details. The ideas behind the implementation are detailed in the \ref{sec:dsl} \nameref{sec:dsl} chapter.

\section{Top level}

The main top level keyword is the `server', which declares a new agent. An agent is basically a web server which can create reports and implement Kubernetes admission webhooks. It is called server, because a web server will be generated from it. It takes 1 argument, its unique name, which must be in kebab-case\footnote{kebab-case: \url{developer.mozilla.org/en-US/docs/Glossary/Kebab_case}}, and opens a block. To be precise it takes 2 arguments, the second argument is a lambda, but in this context it is easier to refer to the last lambda argument of the server function as the `block of the server keyword', or the `server block'.

From now on, I will use this terminology whenever referring to the last lambda argument of a function.

For the code to compile, we also need to import the DSL library, and define a package. The \ref{code:server_boilerplate} code snippet shows the basic template of an agent.

\begin{lstlisting}[caption={Template of a DSL file},language=Kotlin,label=code:server_boilerplate]
package me.btieger

import me.btieger.dsl.*

val serverName = server("server-name") {

}
\end{lstlisting}

Other imports, like the \emph{kotlinx.serialization.json.*} or \emph{io.fabric8.kubernetes.api.model.*} might also be necessary, add them if needed. Use an IDE for Kotlin to automatically add imports.

Inside the server block we get access to three new keywords: `clusterRole', `report', `webhook'

The `clusterRole' is a \emph{String} property. With it, we can assign an existing \emph{ClusterRole} to the agent. The \emph{ClusterRoleBinding} will be automatically generated by the Konstrainer-Core when the agent is deployed. If the agent sends any request to the Kubernetes API, it needs to have authorization to perform that action. With the Konstrainer helm chart a \emph{ClusterRole} is installed too, which gives reed access to all Kubernetes resources. We can reference this \emph{ClusterRole} with the `ReadAny' keyword.

Most properties, just like the 'clusterRole' keyword, can only be set once within a block. This ensures clean scripts where no properties are overridden at the end, avoiding confusion for readers.

\begin{lstlisting}[caption={Usage of the clusterRole keyword},language=Kotlin,label=code:clusterrole_usage]
val server1 = server("server1") {
    clusterRole = ReadAny
}
val server2 = server("server2") {
    clusterRole = "my-custom-cluster-role"
}
\end{lstlisting}

Note that, there is a \emph{ClusterRole} in Kubernetes, which gives unlimited access to everything. It is the `cluster-admin' \emph{ClusterRole}. There will be no permission denied errors, if you use it in the script, however I strongly advise against it. I recommend creating a least-privilaged \emph{ClusterRole} for your script.

The `report' keyword opens a new block. In this block we can create a report from the state of cluster. We can only use the report keyword once in each server block.

The `webhook' keyword defines an admission webhook which listens to certain events in the cluster. This can be used to create warnings, modify or reject some actions.

\subfile{content/50_konstrainer/1_report.tex}

\subfile{content/50_konstrainer/2_webhook.tex}

\section{Summary}

The \ref{chap:case_study1}
This chapter introduced all the features of language in detail. With this knowledge I will continue the case study and not just show the capabilities of the platform, but show how to use it in practice.

TODO A diagram for the entire language can be found in the appendices.

TODO diagram of full langurage
\subfile{content/50_konstrainer/lang_tree.tex}


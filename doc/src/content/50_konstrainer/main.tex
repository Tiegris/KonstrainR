\setlength{\parindent}{0pt}
\setlength{\parskip}{0.6em}

%----------------------------------------------------------------------------
\chapter[Constraint language]{Constraint language: Konstrainer DSL}
\label{chap:konst_dsl}
%----------------------------------------------------------------------------

This chapter will introduce my domain-specific language (DSL) created for enforcing Kubernetes constraints. Every keyword and language feature will be discussed with examples. I'll explain how to use the language, and if needed, provide explanations on why to use it in a particular way. However, I will not delve deep into the implementation details. The ideas behind the implementation are detailed in Chapter \ref{sec:dsl}~\nameref{sec:dsl}.

\section{Top level}

The main top level keyword is the `server', which declares a new agent. An agent is essentially a web server capable of creating reports and implementing Kubernetes admission webhooks. It is called named, because a web server is generated from it. The server keyword takes one argument, its unique name, which must be in kebab-case\footnote{kebab-case: \url{developer.mozilla.org/en-US/docs/Glossary/Kebab_case}}, and opens a block. To be precise, it takes two arguments. The second argument is a lambda, but in this context, it is more convenient to refer to the last lambda argument of the `server' keyword as the `block of the server keyword' or the `server block.' I will use this terminology whenever referring to the last lambda argument of a function.

For the code to compile, we also need to import the DSL library, and define a package. The \ref{code:server_boilerplate} code snippet shows the basic template of an agent.

\begin{lstlisting}[caption={Template of a DSL file},language=Kotlin,label=code:server_boilerplate]
package me.btieger

import me.btieger.dsl.*

val serverName = server("server-name") {

}
\end{lstlisting}

Other imports, such as \lstinline|kotlinx.serialization.json.*| or \lstinline|io.fabric8.kubernetes.api.model.*|, might also be necessary. Add them if needed, or use an IDE for Kotlin to automatically add imports.

Inside the server block we get access to three new keywords: `clusterRole', `report', `webhook'.

The `clusterRole' is a \emph{String} property. Using it, we can assign an existing \emph{ClusterRole} to the agent. The \emph{ClusterRoleBinding} will be automatically generated by the Konstrainer-Core when the agent is deployed. If the agent sends any request to the Kubernetes API, it needs authorization to perform those actions. The Konstrainer Helm chart installs a \emph{ClusterRole} that provides read access to all Kubernetes resources. We can reference this \emph{ClusterRole} with the `ReadAny' keyword.

Most properties, such as the `clusterRole' keyword, can only be set once within a block. This ensures clean scripts where properties are not overridden at the end, avoiding confusion for readers.

\begin{lstlisting}[caption={Usage of the clusterRole keyword},language=Kotlin,label=code:clusterrole_usage]
val server1 = server("server1") {
  clusterRole = ReadAny
}
val server2 = server("server2") {
  clusterRole = "my-custom-cluster-role"
}
\end{lstlisting}

Note that, there is a \emph{ClusterRole} in Kubernetes, which gives unlimited access to everything. It is the `cluster-admin' \emph{ClusterRole}. There will be no permission denied errors, if you use it in the script, however I strongly advise against it. I recommend creating a least-privilaged \emph{ClusterRole} for your script.

The main functionality of an agent can be described using the `report' and `webhook' keywords. The `report' keyword opens a new block It can only be used once in each server block. On the other hand, the `webhook' keyword defines an admission webhook that listens to specific events in the cluster. This can be utilized to generate warnings, modify, or reject certain actions.

\subfile{content/50_konstrainer/1_report.tex}

\subfile{content/50_konstrainer/2_webhook.tex}

\section{Summary}

This chapter introduced all the features of the language in detail. With this knowledge I will continue the case study by demonstrating how to write new scripts.

In figures \ref{fig:fltree_1}, \ref{fig:fltree_2} and \ref{fig:fltree_3}, I have included the full tree of the DSL keywords, excluding the `webhookConfigBundle' keyword and some constants. These figures display the scopes of the keywords, where they can be used, their role, type, arguments, and default values.

\subfile{content/50_konstrainer/lang_tree.tex}

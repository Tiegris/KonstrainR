\section{Report block}
\label{sec:report}

The `report' block allows us to create an overview of the state of the cluster relatively easily. First we should understand how any report could be created, irrespective of this DSL. To generate any report, the necessary data must be fetched, processed, and aggregated. To support these use-cases, I created dedicated language features while also leveraging existing libraries and the built-in features of Kotlin.

In my model of creating a report, the first step involves fetching data from the Kubernetes API, such as the list of pods and deployments. Optionally, you can perform additional processing, such as filtering out namespaces you don't care about or extracting pod labels for easier resource associations. Finally, you create aggregation groups where you can tag certain Kubernetes resources based on custom logic. Creating an aggregation group involves iterating over a kind of resource, like deployments or pods, and applying one or more \emph{if} statements. If an \emph{if} statement evaluates to true, a label is attached to that resource in the resulting report. The pseudo-Java code snippet in \ref{code:pseudo_report} illustrates how this model works.

A small note on why I designed the model this way: API calls are orders of magnitude slower than operations made on the CPU. My model minimizes the number of API calls, making the process faster. An alternative approach could be to fetch only the services first and, during the loop, fetch the pods for only that service. This would be slightly more memory-efficient but significantly slower. I approximated the memory footprint of a pod representation to be around 5 KB. A thousand pods would take approximately 5 MB, which is insignificant, but fetching them only once can save us a thousand API calls.

\begin{minipage}{\linewidth}
\begin{lstlisting}[caption={Example report in pseudo-Java},language=Java,label=code:pseudo_report]
KubernetesClient k8s = /* Instantiate a client */;
// Create map to associate our resources with their custom tags
Map<ServicePodPairs, List<String>> aggregationGroup = new HashMap<>();
// Fetching data
List<Service> services = k8s.services().list().getItems();
List<Pod> pods = k8s.pods().list().getItems();

// Associate services with their pods
List<ServicePodPairs> podsWithServices = /* Custom logic */

for (ServicePodPairs pair : podsWithServices) {
  // Create an empty list for the pair, to store its tags.
  aggregationGroup.set(pair, new ArrayList<String>);
  if (pair.getPods().size().equals(0)) {
    aggregationGroup.get(pair).add("Service has no pods");
  }
  if (pair.getService() == null) {
    aggregationGroup.get(pair).add("No service for the pods");
  }
}
\end{lstlisting}
\end{minipage}

\subsection{Fetching data from K8s}

For fetching data, I've created two language features: the imperative `kubectl' and declarative `kubelist' commands. Both commands open a new block in which we can access the Kubernetes API. Inside the block of these commands, everything is executed on an instance of a KubernetesClient class from the fabric8\footnote{fabric8: https://github.com/fabric8io/kubernetes-client} library. This is an open-source third-party Kubernetes client for Java. In Kotlin terminology, we say that the receiver of the lambda function is the KubernetesClient class.

Both commands are exception-safe, meaning if an error occurs inside their block or during the Kubernetes API call, they return a default value and include the error in the report instead of causing the entire report to fail. The default return value for `kubelist' is an empty list, for `kubectl' it is null. They are also null-safe. `kubelist' always returns a list of non-null values, and the list itself is never null. Although `kubectl' can return a null value, it is marked nullable, so the built-in null safety of Kotlin ensures that it is used only safely.

The example in \ref{code:kubelist_usage} shows how to get the list of pods from all user-created namespaces.

\begin{lstlisting}[caption={Usage of kubelist and kubectl},language=Kotlin,label=code:kubelist_usage]
val pods: List<Pod> = kubelist(omittedNss = nonUserNss) { pods() }
val pods: List<Pod>? = kubectl { 
    pods().inAnyNamespace().list().items 
  }?.filter { 
    it.metadata.namespace !in nonUserNss 
  }
\end{lstlisting}

Note that explicitly specifying the type is not necessary in this case; the compiler can infer the type. However, there are other language features that require explicit type specification.

The `kubelist' is more convenient and streamlined, and I recommend using it primarily. On the other hand, `kubectl' provides full control, which might be necessary in some edge cases.

With `kubectl', the return value of the command will be the return value within its scope, or null if an error occurs. While you have full control, you have to handle everything manually. The best use case for `kubectl' is when you only want to fetch a single resource and not the list of resources, since `kubelist' can only fetch lists of resources. The code snippet in \ref{code:get_logs} shows how to get the logs of all pods, which can only be done using the `kubectl' keyword.

\begin{lstlisting}[caption={Download pod logs},language=Kotlin,label=code:get_logs]
val pods = kubelist { pods() }
val logs: Map<Pod, String?> = pods.associateWith {
  kubectl {
    pods()
      .inNamespace(it.metadata.namespace)
      .withName(it.metadata.name)
      .log
  }
}
\end{lstlisting}

A small note on the `kubectl' command: Since it provides full control to the Kubernetes API, we can send requests other than GET. For example, we can express in our report that we want to delete something, but the command will hopefully fail due to authorization issues. Executing commands that modify the cluster would be ill-advised because a report runs whenever the `/aggregator' endpoint of the agent is called. Modifying the cluster every time someone refreshes the report page in their browser is certainly not a desired behavior.

With `kubelist', many things are added automatically. In the \ref{code:kubelist_usage} code snippet you can see that the `.inAnyNamespace().list().items' part is omitted because it is added automatically. Let's break down what this part actually does. When executing Kubernetes API calls, you need to specify the namespace you are working in. Without it, the default value would be the namespace of the agent. That's why the `inAnyNamespace()' part is added, because when making a report, we want to work with multiple namespaces. The `list()' function executes the actual GET request. The `items' property is derived by the Kotlin language from the `getItems()' Java function. The `list()' function returns a \emph{ResourceList} object, and the `getItems()' returns the resources as a `List<Resource>' object.

The filtering of user namespaces is also approached differently. With the `kubectl' command, filtering is done in a more imperative way by defining how to exactly ignore those namespaces. On the other hand, with `kubelist' we only have to set a parameter.

Let's go through the namespace selection options of the `kubelist' command. The \ref{code:kubelist} code snippet shows all the possible ways to use this command.

\begin{lstlisting}[caption={Usages of kubelist},language=Kotlin,label=code:kubelist]
val pods = kubelist { pods() }
val pods = kubelist(namespace = "my-namespace") { pods() }
val pods = kubelist(omittedNss = listOf("kube-system")) { pods() }
\end{lstlisting}

With the first method, it lists the pods from all namespaces except the system namespaces.

With the second method, you can select a single namespace to work within.

The last method demonstrates how to override the list of ignored namespaces. There are three built in lists: none, nonUserNss, and systemNss. The \ref{code:kubelist_lists_impls} code snippet shows which namespaces they include. Using \lstinline|kubelist(omittedNss = systemNss) { pods() }| is equivalent to \lstinline|kubelist { pods() }| since systemNss is the default for `kubelist'.

\begin{minipage}{\linewidth}
\begin{lstlisting}[caption={Namespace list constants},language=Kotlin,label=code:kubelist_lists_impls]
val systemNss = listOf("kube-system", "kube-node-lease", "kube-public")
val nonUserNss = listOf(
  "kube-system", "kube-node-lease", "kube-public", "default"
)
val none: List<String> = listOf()
\end{lstlisting}
\end{minipage}

\subsection{Transforming data}

Sometimes, you may want to process the data between fetching it and creating an aggregation. Use-cases for this include filtering based on certain criteria or associating different types of resources, such as pods with services. For example, if you want to associate pods with other resources, it's beneficial to extract the pod labels to a set. This makes it easier to write aggregations later. The code snippet in \ref{code:pod_labels} demonstrates how to extract pod labels and use them to determine if a service has any pods.

\begin{lstlisting}[caption={Extraction of pod labels},language=Kotlin,label=code:pod_labels]
val pods = kubelist { pods() }
val podLabels = pods.map { it.metadata.labels }.toHashSet()
val service = /* Get a single service resource */
val serviceHasPods = podLabels.any { podLabel ->
  service.spec.selector.all { podLabel.entries.contains(it) }
}
\end{lstlisting}

It is complex, and I admit it. However, this complexity is inherent at the mathematical level. Expressing the formula that determines whether a service has pods involves intricate mathematical precision, requiring first-order logic. This complexity can be challenging, especially in the context of the already intricate Kubernetes infrastructure.

For data manipulation and resource associations, I didn't introduce anything new; instead, I relied on the extension methods of collections from the Kotlin standard library\footnote{https://kotlinlang.org/docs/collection-transformations.html}}. These functions are excellent for data processing, and I extensively used them during the creation of the case study.

\subsection{Aggregation groups}

This is the part of the language which allows to create the end result of the report. Behind the scenes, it works in such a way, that it iterates over a list of resources and applies a condition to each element of it. If a condition is true to an element, a label is attached to that element.

To create an aggregation group, use the `aggregation' keyword, like seen in the \ref{code:aggregation_group} example. It has two required arguments: the name of the aggregation group and the list of resources you are tagging. This keyword also opens a new block, where the tagging can performed.

\subsection{Aggregation Groups}

To create an aggregation group, use the `aggregation' keyword, as seen in the \ref{code:aggregation_group} example. It has two required arguments: the name of the aggregation group and the list of resources you are tagging. This keyword also opens a new block where the tagging can be performed.

\begin{minipage}{\linewidth}
\begin{lstlisting}[caption={Aggregation group example},language=Kotlin,label=code:aggregation_group]
aggregation("Pods", pods) {
  tag("No security context") {
    item.spec.securityContext == null
  }
}
\end{lstlisting}
\end{minipage}

There is an extra optional argument, the `tagKey' argument, which determines how to identify a resource in the end result. In Kubernetes, any resource can be uniquely identified by the tuple of its name, namespace, and full resource name, which is the combination of the resource kind and apiVersion. The default implementation of the `tagKey' checks if the list item is a Kubernetes resource. If it is, it extracts the aforementioned information from the item. If the item is not a Kubernetes resource, it checks if it is a map entry with the key being a Kubernetes resource and tries to extract the information from the key of the map entry. If it's neither of these, an exception is thrown. The example in \ref{code:tagkey} shows how to override the default `tagKey' implementation.

\begin{lstlisting}[caption={Override tagKey},language=Kotlin,label=code:tagkey]
aggregation("Services", services, tagKey = {
  TagMeta(
    item.fullResourceName,
    item.metadata.name,
    item.metadata.namespace
  )
}) {
  /*...*/
}
\end{lstlisting}

I recommend using the named argument syntax, but it is not necessary. The value for the argument must be a function that returns a \emph{TagMeta} object. Use the `item' keyword inside your lambda to reference a general element of the list.

Within the body of the aggregation block, we gain access to the `tag' keyword. Its first argument is a label that gets attached to the resource. The block of the `tag' keyword is a boolean expression, representing the condition which is applied to each item.

To write the complex boolean expressions of tag conditions, I recommend using extension functions from the Kotlin standard library. Unfortunately, these boolean expression can get very complicated, because Kubernetes itself is complex. Additionally, not all Kubernetes resource fields are implemented in the same way. Some have \emph{null} as default value, some have empty string, while others have an empty object or empty list. To determine whether to use ` == null' or `.isEmpty()' when looking for default or undefined values (e.g.: not defined readiness probe), refer to the Kubernetes documentation, employ the method of trial and error, or inspect a test resource using the `kubectl get <resource name> -o json' CLI command.

Note that the label does not actually get attached to the resource, for example in the form of an annotation. Therefore, no modifications are made to the cluster. The tagging only exists in the report, and the `tagKey' function is used to derive a human-readably and unique ID for the resource.

\subsection{Error handling}

During the execution of the report block, many errors may occur. If an exception occurs in the middle of the execution, the rest of the code would not run, preventing the report from generating. To circumvent this, we could wrap every line of code in a try-catch block, but that would compromise the readability and sleek appearance of the language. Therefore, I implemented a more elegant solution.

My main idea was: why wrap everything once more in another block, the try-catch block, when everything is already in the block of some language feature? Let's contain the exceptions by executing the blocks of the language features in a try-catch block. As an example, see the implementation of the `kubectl' keyword on the \ref{code:error_handling} code snippet.

\begin{lstlisting}[caption={kubectl error handling implementation},language=Kotlin,label=code:error_handling]
fun <T> kubectl(block: KubernetesClient.() -> T): T? {
  return try {
    _kubectl.run(block)
  } catch (e: Exception) {
    _errors += "Error during kubectl call: ${e.message}"
    null
  }
}
\end{lstlisting}

If the command runs without errors, the result is returned. Otherwise, the error message is appended to the list of errors, which is attached to the end of the report. Despite the error, the rest of the report continues to generate. This error report can be used to debug the DSL script.

By implementing many language features like this, most errors can be contained automatically in a small area. At the time of finishing my thesis, only the `kubectl', `kubelist', and `aggregation' keywords implement this behavior. It could be a future improvement to further decrease the impact of errors.

In the continuation of the case study, there will be an example of error handling.

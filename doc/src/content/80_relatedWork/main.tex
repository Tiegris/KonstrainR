\setlength{\parindent}{0pt}
\setlength{\parskip}{0.6em}

%----------------------------------------------------------------------------
\chapter{Related works}
\label{chap:relatedWorks}
%----------------------------------------------------------------------------

This chapter will introduce some similar project and solutions to my work.

\section{Open Policy Agent}

I've heard about the existence of the Open Policy Agent when I was almost finished with the Konstrainer platform, but when I researched it for this chapter, I had to realize that OPA and Konstrainer are basically the same software. They provide the solution to the same problems, with a very similar approach. This section is a comparison between the two projects.

Even dough OPA is an open source project started in 2015, with many active developers and around 400 contributors, there are features in my project that are better, but obviously OPA has more features, they could put more focus on security and over all OPA is more polished.

OPA can declare and enforce policies not just for Kubernetes, but for a variety of platforms and tools, like Terraform and SQL. OPA is better in the sense that one tool can be used to manage everything from the infrastructure level, through the platform level, to the application level. On the other hand, being specific to one platform allows my DSL to be more suited for managing the one platform it is designed for.

Sidenote: what OPA calls `policies', I call them `constraints', but they are essentially the same thing.

My DSL is in form of a JVM library, so it is basically an extension to existing JVM languages, mainly Kotlin. On the other side, OPA has created a language from scratch, called \emph{Rego}\footnote{Rego: \url{www.openpolicyagent.org/docs/latest/policy-language/}}, however their platform can also be used as a \emph{Go} library. 

Creating a new language is better in some sense, because there are no limitations, and you can tailor your DSL to your use-case the best. In contrast, it has many drawbacks, because it is hard to use existing libraries and tools of existing languages. You have to implement them on your own. A parser, transpiler or compiler also needs to be implemented. What is more, learning an absolutely new and unique language can be harder for the user, because there are few things they can relate to, and there are fewer documentation and tutorials available on the topic.

Even dough it is easier to learn just a library, in my experience people rather learn something totally new, then learn a library for a foreign programming language. The explanation might be that if people know absolutely nothing about a language or platform, their curiosity drives them to explore and learn. Yet, when they encounter an existing language the only thing they know for sure, is that they don't know it. Kotlin also has a negative psychological effect, because it is mainly used for Android development and engineers might think that the language for Android development can not be used for Kubernetes.

How you develop your policies or constraints with the two platform is also different. For OPA, there is a very nice online playground and VS Code extension. Both can be used to effectively develop and debug your scripts. My DSL requires a heavy-duty IDE, like IntelliJ IDEA. Debugging my DSL scripts without deploying the script to a live cluster is also harder. You need to wrap your script execution in a function which reads an input file from the disk, and call the proper function of the compiled server script object.

Both OPA and Konstrainer uses the same concept for Kubernetes policy enforcement. On the language level OPA hides the k8s specific admission controllers a bit better, than my DSL, but under the hood both uses admission controller webhooks.

The strongest advantage of my platform is the automation of deployment. In both cases, the policy platform can be easily deployed using a couple of commands and a helm chart, however how you deploy a policy is much different. In the Kubernetes tutorial of OPA~\cite{OPA} it is described how to deploy a policy. With OPA you have to manually manage the TLS related keys and certificates required for deploying admission webhooks. Furthermore, the Kubernetes resources like the \emph{Service}, \emph{ClusterRoleBinding}, \emph{ValidatingWebhookConfiguration} and the \emph{Deployment} of the agent must be created by hand. Konstrainer automates these steps, to the extent that deploying a script is just a drag \& drop on the UI. It is worth mentioning that OPA also offers a simplified deployment method using Gatekeeper\footnote{\url{github.com/open-policy-agent/gatekeeper}}. Gatekeeper introduces constraints as Kubernetes custom resources. With Gatekeeper, deploying a policy is just creating a Kubernetes resource which includes the OPA Rego script as a string.

OPA also has a reporting feature. What is much better in OPA's reporting feature is that the same language can be used to validate events and see which already created resources violate the policies. This design is better, than the separate `report' and `webhooks' found in Konstrainer, because you do not need to write basically the same constraints twice.

Both OPA and Konstrainer have some built in policies which can be used out of the box. However, OPA has more since it is much larger project.

In summary OPA is a much larger ecosystem, it supports more platforms, has more features, and it is more polished. On the other hand it is much easier to deploy and manage policies using Konstrainer.

\section{Kyverno}

Kyverno describes itself on their GitHub page like so:
`Kyverno is a policy engine designed for Kubernetes. It can validate, mutate, and generate configurations using admission controls and background scans. Kyverno policies are Kubernetes resources and do not require learning a new language.'~\cite{Kyverno}

Kyverno takes on interesting approach to the problem. Instead of creating a language, it uses custom Kubernetes resources to describe policies. However, to describe k8s resources, the best way is to create YAML files. These YAML files are specific to describing policies, so basically Kyverno uses a YAML based DSL.

Kyverno can essentially do the same as OPA and Konstrainer, but it has an additional feature. Kyverno can also be used to create or delete resources. Konstrainer can do that too, however I do not think that policy enforcer tools should automatically delete resources, so I advise against it. In the Kyverno quick start documentation~\cite{KyvernoQuick}, there is an example which shows how to create image pull secrets in all newly created namespaces. Konstrainer can also do this, but I didn't intend this to be a primary use-case.

Just like OPA, Kyverno also has an online playground which makes development easy. Deploying policies is straightforward as well, it is just a `kubectl apply' command. Since it uses CLI commands instead of a Web UI, it is simpler to integrate it into CI/CD and to use GitOps. Integrating Konstrainer into a CI/CD pipeline is possible, however it would take some \emph{curl} calls.

The grates advantage of Kyverno over my Konstrainer is the large available pre-made policy base. On the \url{https://kyverno.io/policies/} website, there were almost 300 policies at the time this thesis was written.

The most significant advantage of Konstrainer over Kyverno, is its readability. For instance, while Kyverno supports patches, it adheres strictly to the JSON Patch standard, lacking an easily accessible abstraction layer. In contrast, Konstrainer introduces an abstraction layer in such cases. Additionally, creating `joins', when you want to create an association between two or more resources, seem much harder to read. Kyverno also employs special operators that may be challenging to understand without referring to the documentation. In contrast, Konstrainer utilizes straightforward English keywords, making the process much easier to comprehend.

\section{Kubernetes reporting tools}

There are tools which can only create reports but not enforce policies. These tools mostly focus on metrics, like CPU requests, usage and limits. To give an example, the Kubernetes Resource Report project from Henning Jacobs on \url{https://codeberg.org/hjacobs/kube-resource-report} can create an HTML report from the resource requests and usages.

I did not find any other mentionable project which can create reports that are more advanced than just simple metric request and usage reports, although it is possible to create advanced reports manually. The \emph{kubectl} CLI tool alongside with \emph{jq} can be used to get the list of Kubernetes resources which meet a criterion. This is fine if you only want to find violations of one simple policy, but for complex reports which can be generated at anytime it is not the best. The result will be in text format, what is more it is hard to handle errors in bash, so the edge cases will be handled poorly. Therefore, it is good idea to use something more sophisticated for writing complex reports.

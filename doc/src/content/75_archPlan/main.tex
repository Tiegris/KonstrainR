\setlength{\parindent}{0pt}
\setlength{\parskip}{0.6em}

%----------------------------------------------------------------------------
\chapter{Implementation}
\label{chap:archPlan}
%----------------------------------------------------------------------------

\subfile{content/75_archPlan/dsl.tex}

\subfile{content/75_archPlan/arch.tex}

\subfile{content/75_archPlan/process.tex}


\section{Packaging}
how the agent and core containerised
explain the helm chart installing them



\section{Security}
security:
    passwds
        basic auth
    TLS for core server 



\section{Evaluation}

\subsection[Chosen technology]{Chosen technology: Do I recommend Ktor?}

Since I chose Kotlin as the primary technology for my DSL, I needed to select a web server framework that supports Kotlin to easily serve the DSL scripts. Spring Boot would have been the least risky choice, as it is a mature, production ready, and widely used framework with numerous pre-existing libraries. The alternative option was Ktor, a relatively new framework built from the ground up using Kotlin and Coroutines. It claims to be lightweight and flexible. Despite having prior experience with Spring Boot, I wanted to try out Ktor. This thesis was also an experiment with the Ktor framework, and this section reflects my opinion on this new framework.

This experiment was quite exciting, I have learned a lot from it. In my experience, it is relatively easy to create basic things, such as a REST API with authentication and persistent storage. However, it becomes significantly more challenging when you aim to integrate authorization with authentication or when you want to make your code testable. While there is sufficient documentation on the basics, there are no guidelines or best practices for the more advanced use cases, and the examples I mentioned are not even that advanced. Since this is a new framework, the community is small, so there are no forums or blogs on best practices. Many times, I had to figure out on my own how to use the framework effectively.

For SQL persistence, Ktor recommends the Exposed\footnote{\url{https://github.com/JetBrains/Exposed}} ORM library, a lightweight SQL library built on top of JDBC driver for the Kotlin language. Over all, it is a good library, but just like with the core Ktor, figuring out the best practices was challenging. It was particularly difficult when I aimed to make the code testable, as dependency injection was never mentioned in any of the tutorials; instead, global variables were used. I also found that Exposed is not always reliable. When updating Ktor versions, I encountered runtime errors whit previously working code. I had to fix the function responsible for reading the compiled DSL script from the database three times.

For dependency injection (DI), I utilized Koin\footnote{\url{https://insert-koin.io/}}, which is a quite good DI framework. It's unfortunate that it is not emphasized in the core Ktor documentation. It is yet another sign that Ktor needs more time to mature.

In conclusion, I would not recommend using Ktor for enterprise purposes due to the lack of guidelines, reliability issues, and missing features, which make it too risky for business applications. On the other hand, it is easy to learn and use, and enjoyable to work with. Therefore, for hobby projects, it can be a good idea, and perhaps in a couple of years, it will be mature enough for business applications as well.

In the case of my thesis, it probably would have been wiser to use Spring Boot instead, but the exploration was worthwhile. I do not regret choosing Ktor.

\subsection[Design Reflections]{Design Reflections: What could have been better}

Looking back at the finished project, I can see some things that I could have done better. I'm not referring to the features I didn't have enough time to polish, but rather the aspects that I implemented in a suboptimal way, and fixing them would require a significant effort. This section is dedicated to these design oversights.

The first design oversight is that I essentially created two languages: one for creating admission webhooks and another for generating reports. Instead, it would have been better to create a unified language to describe constraints, because for every constraint, we would like to enforce it and list all the violations simultaneously. My solution is more flexible and might prove useful in some rare use cases; however, for the general use case, it would have been better to have one unified language.

The second suboptimal design decision concerns how a script is served. With the current implementation, there is one Ktor server running for each deployed script, even though a single agent could handle the load of multiple running scripts. This setup can negatively impact resource allocations, as one agent receives very little load, but the minimal memory requirements of a Ktor server are relatively high, around 128 MiB. If necessary, the issue can be mitigated by joining multiple scripts together. But to be honest, if a company is large enough to need something like Konstrainer, wasting a couple of hundred megabytes of RAM is not likely to cause significant problems.

These were the mayor design oversights, which would take a lot of work to fix. The remaining imperfect aspects are more about being unpolished than badly designed. Aspects such as the non-reactive UI or the CI/CD integration could be easily improved over time, but I had to set a limit on the scope.

\subsection{Performance}

It takes around 40 seconds to compile a script

it is blazing fast
one agent serves only one script, this could be improved, but noproblem, an agent is relatively lightweight


%----------------------------------------------------------------------------
\appendix
%----------------------------------------------------------------------------
\chapter*{\fuggelek}\addcontentsline{toc}{chapter}{\fuggelek}
\setcounter{chapter}{\appendixnumber}
%\setcounter{equation}{0} % a fofejezet-szamlalo az angol ABC 6. betuje (F) lesz
\numberwithin{equation}{section}
\numberwithin{figure}{section}
\numberwithin{lstlisting}{section}
%\numberwithin{tabular}{section}

%----------------------------------------------------------------------------
\section{Case study resources}
%----------------------------------------------------------------------------
\begin{lstlisting}[caption={initial-setup.sh},language=bash,label=appendix:csr:setup]
#!/bin/bash
set -e

# Setup
home=$(pwd)

if [ "$BUILD_IMAGES" = "1" ]; then
  # Build images
  repo=${REPO:-tiegris}
  cd $home/apps/items
  docker build -t $repo/apples-items:latest .
  docker push $repo/apples-items:latest
  cd $home/apps/users
  docker build -t $repo/apples-users:latest .
  docker push $repo/apples-users:latest
  cd $home/apps/accounting
  docker build -t $repo/apples-accounting:latest .
  docker push $repo/apples-accounting:latest
else
  echo "Skipping build images"
fi


# Initial creation of users app
kubectl create ns users || true
kubectl apply -n users -f $home/k8s/users.yaml

# Initial creation of items app
kubectl create ns items || true
kubectl apply -n items -f $home/k8s/items.yaml

# Simulate a history of changes
export v
for v in {1..10}; do
  yq eval 'select(.kind == "Deployment").spec.template.metadata.annotations.v = env(v)' $home/k8s/items.yaml -i
  kubectl apply -n items -f $home/k8s/items.yaml
done
# Restore yaml file
yq eval 'select(.kind == "Deployment").spec.template.metadata.annotations.v = "0"' $home/k8s/items.yaml -i

# Simulate John Athan creating test namespace
kubectl create ns jathan-test || true
# Simulate Ida Red creating test namespace and leaving some leftover junk in it
kubectl create ns ired-test || true
kubectl apply -n ired-test -f $home/k8s/junk.yaml

# install mongodb
# https://bitnami.com/stack/mongodb/helm
kubectl create ns mongo
helm install -n mongo my-release oci://registry-1.docker.io/bitnamicharts/mongodb

export MONGODB_ROOT_PASSWORD=$(kubectl get secret --namespace mongo my-release-mongodb -o jsonpath="{.data.mongodb-root-password}" | base64 -d)
export MONGO_HOST="my-release-mongodb.mongo.svc.cluster.local"

yq eval 'select(.kind == "Deployment").spec.template.spec.containers[0].env[0].value = env(MONGODB_ROOT_PASSWORD)' $home/k8s/items.yaml -i
yq eval 'select(.kind == "Deployment").spec.template.spec.containers[0].env[1].value = env(MONGO_HOST)' $home/k8s/items.yaml -i
kubectl apply -n items -f $home/k8s/items.yaml
\end{lstlisting}

\begin{lstlisting}[caption={accounting.yaml},language=YAML,label=appendix:csr:accounting]
apiVersion: apps/v1
kind: Deployment
metadata:
  name: accounting
spec:
  selector:
    matchLabels:
      app: accounting
  template:
    metadata:
      labels:
        app: accounting
    spec:
      containers:
        - name: accounting
          image: tiegris/apples-accounting:latest
          ports:
            - containerPort: 8079
\end{lstlisting}

\begin{lstlisting}[caption={items.yaml},language=YAML,label=appendix:csr:items]
apiVersion: v1
kind: Service
metadata:
  name: items
spec:
  selector:
    app: items
  ports:
    - port: 8080
      targetPort: 8079
---
apiVersion: apps/v1
kind: Deployment
metadata:
  name: items
spec:
  selector:
    matchLabels:
      app: items
  template:
    metadata:
      labels:
        app: items
      annotations:
        v: "0"
    spec:
      containers:
        - name: items
          env:
            - name: MongoPass
              value: cLJBxAPrnY
            - name: MongoHost
              value: my-release-mongodb.mongo.svc.cluster.local
          image: tiegris/apples-items:latest
          resources:
            limits:
              memory: "128Mi"
              cpu: "500m"
          ports:
            - containerPort: 8079
---
apiVersion: v1
kind: PersistentVolumeClaim
metadata:
  name: items-pvc
spec:
  resources:
    requests:
      storage: 50M
  volumeMode: Filesystem
  accessModes:
    - ReadWriteOnce
\end{lstlisting}

\begin{lstlisting}[caption={users.yaml},language=YAML,label=appendix:csr:users]
apiVersion: v1
kind: Service
metadata:
  name: users
spec:
  selector:
    app: users
  ports:
  - port: 8080
    targetPort: 8079
---
apiVersion: apps/v1
kind: Deployment
metadata:
  name: user
spec:
  selector:
    matchLabels:
      app: user
  template:
    metadata:
      labels:
        app: user
    spec:
      containers:
      - name: user
        image: tiegris/apples-users:latest
        resources:
          limits:
            memory: "128Mi"
            cpu: "500m"
        ports:
        - containerPort: 8079
        volumeMounts:
          - mountPath: "/app/data"
            name: user-storage
      volumes:
      - name: user-storage
        persistentVolumeClaim:
          claimName: users-pvc
---
apiVersion: v1
kind: PersistentVolumeClaim
metadata:
  name: users-pvc
spec:
  resources:
    requests:
      storage: 50M
  volumeMode: Filesystem
  accessModes:
    - ReadWriteOnce


\end{lstlisting}

\begin{lstlisting}[caption={junk.yaml},language=YAML,label=appendix:csr:junk]
apiVersion: v1
kind: Pod
metadata:
  name: my-test-2
  labels:
    app: my-test-2
spec:
  containers:
  - name: myapp
    image: "there-is-no-way-an-image-exists-with-this-tag-on-docker.io"
    command: ["sleep", "infinity"]
    resources:
      limits:
        memory: "128Mi"
        cpu: "500m"
    ports:
      - containerPort: 8443
---
apiVersion: v1
kind: Pod
metadata:
  name: my-test
  labels:
    app: my-test
spec:
  containers:
  - name: myapp
    image: alpine
    command: ["sleep", "infinity"]
    resources:
      limits:
        memory: "128Mi"
        cpu: "500m"
    ports:
      - containerPort: 8443
\end{lstlisting}

\begin{lstlisting}[caption={test-policy.yaml},language=YAML,label=appendix:csr:testpolicy]
apiVersion: apps/v1
kind: Deployment
metadata:
  name: policy-test
spec:
  selector:
    matchLabels:
      app: policy-test
  template:
    metadata:
      labels:
        app: policy-test
    spec:
      containers:
        - image: alpine
          name: policy-test
          command: ["sleep", "infinity"]
\end{lstlisting}

\begin{lstlisting}[caption={readPods.yaml},language=YAML,label=appendix:csr:readpods]
apiVersion: rbac.authorization.k8s.io/v1
kind: ClusterRole
metadata:
  name: read-pods
rules:
  - apiGroups: [""]
    resources: ["pods"]
    verbs: ["get", "watch", "list"]
\end{lstlisting}

%----------------------------------------------------------------------------
\clearpage\section{Case study DSL scripts}
%----------------------------------------------------------------------------

\begin{lstlisting}[caption={BasicDiagnostics.kt},language=Kotlin,label=appendix:csr:bdkt]
package me.btieger.builtins

import me.btieger.dsl.*

val diagnosticsServer = server("basic-diagnostics") {
  clusterRole = ReadAny

  report {
    val pods = kubelist { pods() }
    val podLabels = pods.map { it.metadata.labels }.toHashSet()

    val services = kubelist { services() }
    aggregation("Services", services) {
      tag("Has external IP") {
        item.spec.externalIPs.isNotEmpty()
      }
      tag("No backend") {
        podLabels.none { podLabel ->
          item.spec.selector.all { podLabel.entries.contains(it) }
        }
      }
    }

    val deployments = kubelist { apps().deployments() }
    aggregation("Deployments", deployments) {
      tag("No resources defined") {
        item.spec.template.spec.containers.any { 
          it.resources.limits.isNullOrEmpty() || it.resources.requests.isNullOrEmpty() 
        }
      }
      tag("No probes") {
        item.spec.template.spec.containers.any { it.livenessProbe == null }
      }
      tag("Has long history") {
        item.spec.revisionHistoryLimit > 4
      }
      tag("No node affinity") {
        val podSpec = item.spec.template.spec
        podSpec.affinity == null && 
        podSpec.nodeSelector.isNullOrEmpty() && 
        podSpec.nodeName == null
      }
    }

    val pvcs = kubelist { persistentVolumeClaims() }
    aggregation("PVCs", pvcs) {
      tag("Unused") {
        pods.none { pod ->
          pod.metadata.namespace == item.metadata.namespace &&
          pod.spec.volumes.any { 
            volume -> volume.persistentVolumeClaim?.claimName == 
              item.metadata.name 
          }
        }
      }
    }

    val nss = kubelist(omittedNss = none) { namespaces() }
      .filter { it.metadata.name !in nonUserNss }
    aggregation("Namespaces", nss) {
      tag("Has no pods") {
        pods.none { pod -> pod.metadata.namespace == item.metadata.name }
      }
    }

    aggregation("Pods", pods) {
      tag("Not running or not succeeded") {
        item.status.phase != "Running" && item.status.phase != "Succeeded"
      }
      tag("Image pull backoff") {
        item.status.containerStatuses.any { it.state.waiting?.reason == "ImagePullBackOff" } ||
        item.status.containerStatuses.any { it.state.waiting?.reason == "ErrImagePull" }
      }
      tag("Dangling pod") {
        item.metadata.ownerReferences.isEmpty()
      }
      tag("Not Running") {
        item.status.containerStatuses.any { it.state.running == null }
      }
      tag("No security context") {
        item.spec.securityContext == null
      }
    }
  }
}
\end{lstlisting}

\clearpage

\begin{lstlisting}[caption={BasicEnforce.kt},language=Kotlin,label=appendix:csr:bekt]
package me.btieger.builtins

import io.fabric8.kubernetes.api.model.LabelSelectorRequirement
import kotlinx.serialization.json.JsonNull
import me.btieger.dsl.*

val whcb = webhookConfigBundle {
  operations(CREATE, UPDATE)
  apiGroups(APPS)
  apiVersions(ANY)
  resources(DEPLOYMENTS, STATEFULSETS, DAEMONSETS)
  namespaceSelector {
    matchLabels = mapOf(
      "managed" to "true"
    )
  }
  failurePolicy(FAIL)
  logRequest = true
}

val webhookServer = server("basic-enforce") {
  clusterRole = ReadAny

  webhook("warn-no-security-context", whcb) {
    behavior {
      warnings {
        if (request jqx "/object/spec/template/spec/securityContext" == JsonEmpty)
          warning("No security context")
      }
    }
  }

  webhook("node-affinity", whcb) {
    behavior {
      val podSpec = (request jqx "/object/spec/template/spec")
      val affinity = podSpec jqx "affinity"
      val nodeSelector = podSpec jqx "nodeSelector"
      val nodeName = podSpec jqx "nodeName"
      allowed {
        !(affinity is JsonNull && nodeSelector is JsonNull && nodeName is JsonNull)
      }
      status {
        message = "Deployment must have some kind of node affinity! (affinity, nodeSelector, nodeName)"
      }
    }
  }
  webhook("cut-history", whcb) {
    behavior {
      val revisionHistoryLimit = (request jqx "/object/spec/revisionHistoryLimit" parseAs int) ?: 10
      warnings {
        if (revisionHistoryLimit > 4) 
          warning("RevisionHistoryLimit was set to 4, from original: $revisionHistoryLimit")
      }
      patch {
        if (revisionHistoryLimit > 4) 
          replace("/spec/revisionHistoryLimit", 4)
      }
    }
  }
  webhook("deny-no-resources", whcb) {
    behavior {
      val containers = podSpec?.containers ?: listOf()
      allowed {
        containers.all {
          it.resources?.limits?.isNotEmpty() == true &&
          it.resources?.requests?.isNotEmpty() == true
        }
      }
      status {
        message = "${currentObject?.kind} must have resource definitions!"
      }
    }
  }
  webhook("warn-default-ns") {
    operations(CREATE, UPDATE, DELETE)
    apiGroups(ANY)
    apiVersions(ANY)
    resources(ANY)
    failurePolicy(IGNORE)
    namespaceSelector {
      matchExpressions = listOf(
        LabelSelectorRequirement().apply {
          operator = "In"
          key = "kubernetes.io/metadata.name"
          values = listOf("default")
        }
      )
    }
    behavior {
      var ns = request jqx "/object/metadata/namespace" parseAs string
      if (ns == null) // In case of delete, the object is stored in oldObject.
        ns = request jqx "/oldObject/metadata/namespace" parseAs string
      warnings {
        warning("You are working in the namespace: $ns")
      }
    }
  }
}
\end{lstlisting}
%----------------------------------------------------------------------------
\clearpage\section{Builder script}
%----------------------------------------------------------------------------

\begin{lstlisting}[caption={Builder script},language=bash,label=appendix:builder]
#!/bin/bash

function log() {
  echo "$1" >> $logsfile
}

base_url="https://${KSR_CORE_BASE_URL}:8080/api/v1/dsls"
dsl_id="${KSR_DSL_ID}"
secret="${KSR_SECRET}"

logsfile='/app/logs'
file='/app/framework/lib/src/main/kotlin/me/btieger/DslInstance.kt'
jar='/app/framework/lib/build/libs/lib.jar'

cp "/app/tls-cert/rootCa.pem" /usr/local/share/ca-certificates/rootCa.crt
update-ca-certificates

log "=== BEGIN ERROR REPORT ==="
log "BASE_URL = $base_url"
log "DSL_ID = $dsl_id"

function try_do() {
  wget "${base_url}/${dsl_id}/file" -O "${file}" && log "Download done" || { log "Download failed" && error_report; }
  sed -i 's/package .*/package me.btieger/' "${file}" && log "Sed done" || { log "Sed failed" && error_report; }
  cd /app/framework
  ./gradlew jar >> $logsfile 2>&1 && log "Build done" || { log "Build failed" && error_report; }
  log "Compilation done"
  curl -F filename="${jar}" -F upload=@${jar} -H "Content-Type: application/java-archive" "${base_url}/${dsl_id}/jar" || { log "Upload failed" && error_report; }
}

function error_report() {
  log "=== END ERROR REPORT ==="
  curl -H "Content-Type: text/plain" -X POST --data "$(cat /app/logs)" "${base_url}/${dsl_id}/jar"
  exit 1
}

try_do || error_report
\end{lstlisting}

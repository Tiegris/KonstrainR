\setlength{\parindent}{0pt}
\setlength{\parskip}{0.6em}

%----------------------------------------------------------------------------
\chapter{Preliminaries}
\label{chap:prerequisites}
%----------------------------------------------------------------------------

\section{Kubernetes}

\section{Kotlin}

\subsection{Extension functions}
\label{sec:extension}

TODO reference: https://kotlinlang.org/docs/extensions.html\#extension-functions

An extension function can add new functionality to an existing class without modifying it. In Java, this can be implemented as shown in the \ref{code:lang3} code snippet. Although Java lacks native support for extension methods, there is a workaround using public static utility functions. In contrast, Kotlin supports extension functions natively, as demonstrated in the \ref{code:lang4} code snippet.

\begin{lstlisting}[caption={Extension functions in Java},language=Java,label=code:lang3]
public class MyClass {
    ...
}
public class MyClassExtensions {
    private MyClassExtensions() {}
    public static void myExtensionFunction(MyClass self) {
        self.doSomething();
    }
}
// Invocation:
final var myClass = new MyClass();
myExtensionFunction(myClass);
\end{lstlisting}

\begin{lstlisting}[caption={Extension functions in Kotlin},language=Kotlin,label=code:lang4]
class MyClass {
    ...
}
fun MyClass.myExtensionFunction() {
    this.doSomething()
    // Using the "this" keyword is optional
}
// Invocation:
val myClass = MyClass()
myClass.myExtensionFunction()
\end{lstlisting}

Kubernetes

Kotlin: extension function, receiver, elvis, nullability

DSL

terminology: script, dsl, Konstrainer, block

\section{Domain specific language}
\label{sec:dsl}

This section explains how I created the Konstrainer DSL. I won't explain the details of creating each keyword, as many of them belong to specific categories and share similar patterns within the same category.

\subsection{Basic concepts}

My DSL has three mayor concepts: blocks, properties, and setter functions. The code snippet \ref{code:lang1} explains what they are and provides an example for each.

\begin{lstlisting}[caption={Language concepts},language=Kotlin,label=code:lang1]
// The car keyword opens a new block
car { 
    // The block of the car keyword starts here
    color = "red" // This is a property
    color("red") // This is a setter function
    // The block of the car keyword ends here
}
\end{lstlisting}

Properties and setter functions usually can only be used within the context of a block. This is how it makes sense in the language, for example you can set the color of the car, but it does not make sense to set the color of the top level world. Additionally, the implementation of a block also restricts where other keywords can be used.

The \ref{code:lang1} example can be implemented similarly like in the \ref{code:lang2} code snippet.

\begin{lstlisting}[caption={Basic idea behind a block},language=Kotlin,label=code:lang2]
class CarModel(val color: String)
class CarBuilder {
    var color: String ...
    fun color(value: String) ...
}
fun car(setup: CarBuilder.() -> Unit): CarModel {
    ...
}
\end{lstlisting}

The `car' keyword is implemented as a function with a single lambda argument. In Kotlin, functions are first-class objects, meaning they can be used like any other object. In this example the variable `setup' is an extension function on the CarBuilder class. It takes zero arguments and returns Unit. Extension methods are already explained in the \nameref{sec:extension} section. The Unit return type is also important. It does not imply that the `car' block does not return anything, it only means that the lambda itself does not return anything. However, the `car' block can still return something, as in this case, it returns a CarModel object.

\subsection{Blocks}

\subsubsection{Startup-time evaluated blocks}

Type-safe builders pattern

\subsubsection{Runtime evaluated blocks}

\subsection{`setExactlyOnce' and `setMaxOnce'}

All properties can only be set once, or in the case of setter functions, invoked once. This is to avoid inadvertent overwrites and to enhance script readability. The enforcement of this language constraint is carried out through the `setExactlyOnce' and 7setMaxOnce' property delegates. Kotlin has unique a feature called property delegates, which enables to define a custom behavior to a set of properties. This custom behavior can be added to a library, eliminating the need to copy and paste the behavior into every getter and setter. Instead, you simply utilize the delegate. For detailed information, read the official Kotlin documentation on the topic: \url{https://kotlinlang.org/docs/delegated-properties.html}. Here, I will just show a small example and explain how my custom property delegates work.

\begin{lstlisting}[caption={Lazy getter in Java},language=Java,label=code:lazy0]
class A {
    private String foo;
    public getFoo() {
        if (foo == null) 
            foo = calculateFoo(); // This is a resource intese operation
        return foo;
    }
}
\end{lstlisting}
    
\begin{lstlisting}[caption={Lazy property in Kotlin},language=Kotlin,label=code:lazy1]
class A {
    val foo by lazy(::calculateFoo)
}
\end{lstlisting}

The \ref{code:lazy0} and \ref{code:lazy1} code snippets illustrates how the lazy\footnote{\url{https://en.wikipedia.org/wiki/Lazy_initialization}} behavior can be implemented in Java and in Kotlin. The lazy behavior involves delaying the calculation of a value until it is read, after which it is cached. The Java code snippet shows the issue. When implementing custom behavior for a getter, it must be written each time for every getter with the same behavior. In Kotlin, this behavior can be delegated, making the code not only less bloated but also less prone to copy and paste errors.

\begin{minipage}{\linewidth}
\begin{lstlisting}[caption={setExactlyOnce implementation},language=Kotlin,label=code:setonce]
class setExactlyOnce<T : Any>() : ReadWriteProperty<Any, T> {
    constructor(default: T?) : this() {
        _value = default
    }

    private var _alreadySet = false
    private var _value: T? = null

    override fun getValue(thisRef: Any, property: KProperty<*>): T {
        return _value ?: throw FieldNotSetException(
            "Property ${property.name} not set."
        )
    }

    override fun setValue(thisRef: Any, property: KProperty<*>, value: T) {
        if (_alreadySet)
            throw MultipleSetException(
                "Property ${property.name} can only be set once."
            )
        _alreadySet = true
        _value = value;
    }
}

// Usage:
// A clusterRole can assigned to an agend maximum once
var clusterRole by setMaxOnce<ClusterRoleName>()
// behavior of webhook can only be set exactly once,
// and has no default value
private var behavior: WebhookBehaviorProvider by setExactlyOnce()
// operations of a webhook can only be set exactly once,
// and has a default value
private var _operations: Array<out String>
    by setExactlyOnce(defaults.operations)
\end{lstlisting}
\end{minipage}

The \ref{code:setonce} code snippet demonstrates how the `setExactlyOnce' property delegate is implemented. To fully understand why this seemingly complicated syntax was needed, read the official documentation which explains it in detail. What's important here is, that there is an optional default value which can be set when creating a property. The default value is stored in a private backing field: `\_value', which is null, in case there is no default value. 

When the value is set, the delegate checks the `\_alreadySet' flag. If it is true, an exception is raised; otherwise, it sets the value and the flag. The inclusion of the flag is necessary because a simple null check cannot differentiate between an already set value and a default value.

When the value is read, the delegated checks if the `\_value' is null. If it is, then it raises an exception. However, in the case of the `setMaxOnce' delegate, this check does not exist.

I will not show the implementation of `setMaxOnce' because it closely resembles `setExactlyOnce'.

Error messages are not currently aggregated into a centralized view; instead, they can be found in the logs of either the core component or the agent. A potential future improvement could involve collecting and displaying these error messages as build-time errors.

\subsection{JSON utilities}

The language has unique keywords for JSON parsing. They do not fall into the main categories of blocks and properties, but they are unique. They look more like continuous natural English language. Their implementation is a bit complicated and to understand it, you must first understand some basic concepts of Kotlin.

In Kotlin there are infix functions. You can add the `infix' modifier to an extension function to make it callable using the infix notation. See the \ref{code:infix} code snippet for an example or read the official documentation for more details: \url{https://kotlinlang.org/docs/functions.html#infix-notation}

\begin{lstlisting}[caption={Infix functions},language=Kotlin,label=code:infix]
infix fun Int.plus(that: Int): Int {
    return this + that
}
// Invocation:
val x = 5 plus 6  // Infix notation
val y = 5.plus(6) // This does the same as above
\end{lstlisting}

The second concept you must understand is treating objects as tokens. Objects can be utilized in various ways, but what's important for this section is their ability to create special tokens. You can define two functions with the same name but different arguments. If these arguments are of this kind of tokens, they can be used for pattern matching as shown in the \ref{code:patternobj} code snippet.

\begin{lstlisting}[caption={Pattern matching},language=Kotlin,label=code:patternobj]
object Token1
object Token2

fun foo(o: Token1) {
    ...
}
fun foo(o: Token2) {
    ...
}
// Invocation
val x = Token2
foo(x)      // Runs the second foo function
foo(Token1) // Runs the first foo function
\end{lstlisting}

Combining these Kotlin language features can be used to create fluent, natural language-like constructs. The \ref{code:jqx} code snippet demonstrates how these language concepts are employed to implement the JSON parsing aspects of the Konstrainer DSL.

\begin{lstlisting}[caption={jqx implementation},language=Kotlin,label=code:jqx]
infix fun JsonElement.jqx(selector: String): JsonElement {
    ...
}

object int
object bool
object double
object string

infix fun JsonElement.parseAs(type: int): Int? = 
    nullGuard { /* Function defining how to get JSON field as Int*/ }
infix fun JsonElement.parseAs(type: bool): Boolean? = nullGuard { ... }
infix fun JsonElement.parseAs(type: double): Double? = nullGuard { ... }
infix fun JsonElement.parseAs(type: string): String? = nullGuard { ... }

// Usage:
val x = jsonElement jqx "/foo/bar" parseAs string
\end{lstlisting}

`jqx' is an infix function with a return type of JsonElement. `parseAs' is also an infix function with the type JsonElement as a receiver, so `jqx' and `parseAs' commands can be chained. There is a token created for every the primitive types that can occur in a JSON, and there is a corresponding `parseAs' function for each token. This setup facilitates the extraction of all types from a JSON structure. The `nullGuard' is a private function that return null if the conversion can not happen, for instance, due to a type mismatch or non-existing JSON field. Otherwise, it returns the converted value. The conversion function is defined in its block.

\subsection{Unmarshallers}



\subsection{DslMarker annotation}




\begin{lstlisting}[caption={TODO},language=Kotlin,label=code:todo]

\end{lstlisting}
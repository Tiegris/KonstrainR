\section{Domain specific language}
\label{sec:dsl}

This section explains how I created the Konstrainer DSL. I won't explain how I created each keyword, as most of them fall into specific categories, and every keyword in the same category is created using the same pattern.

\subsection{Basic concepts}

My DSL has three mayor concepts: blocks, properties, and setter functions. The code snippet \ref{code:lang1} explains what they are and provides an example for each.

\begin{lstlisting}[caption={Language concepts},language=Kotlin,label=code:lang1]
// The car keyword opens a new block
car { 
    // The block of the car keyword starts here
    color = "red" // This is a property
    color("red") // This is a setter function
    // The block of the car keyword ends here
}
\end{lstlisting}

Properties and setter functions usually can only be used within the context of a block. This is how it makes sense in the language, for example you can set the color of the car, but it does not make sense to set the color of the top level world. Additionally, the implementation of a block also restricts where other keywords can be used.

The \ref{code:lang1} example can be implemented similarly like in the \ref{code:lang2} code snippet.

\begin{lstlisting}[caption={Basic idea behind a block},language=Kotlin,label=code:lang2]
class CarModel(val color: String)
class CarBuilder {
    var color: String ...
    fun color(value: String) ...
}
fun car(setup: CarBuilder.() -> Unit): CarModel {
    ...
}
\end{lstlisting}

The `car' keyword is implemented as a function with a single lambda argument. In Kotlin, functions are first-class objects, meaning they can be used like any other object. In this example the variable `setup' is an extension function on the CarBuilder class. It takes zero arguments and returns Unit. Extension methods are already explained in the \nameref{sec:extension} section. The Unit return type is also important. It does not imply that the `car' block does not return anything, it only means that the lambda itself does not return anything. However, the `car' block can still return something, as in this case, it returns a CarModel object.


\subsection{Blocks}

\subsubsection{Startup-time evaluated blocks}

Type-safe builders pattern

\subsubsection{Runtime evaluated blocks}

\subsection{`setExactlyOnce' and `setMaxOnce'}



\subsection{JSON utilities}

The language has unique keywords for JSON parsing. They do not fall into the main categories of blocks and properties, but they are unique. They look more like continuous natural English language. Their implementation is a bit complicated and to understand it, you must first understand some basic concepts of Kotlin.

In Kotlin there are infix functions. You can add the `infix' modifier to an extension function to make it callable using the infix notation. See the \ref{code:infix} code snippet for an example or read the official documentation for more details: \url{https://kotlinlang.org/docs/functions.html#infix-notation}

\begin{lstlisting}[caption={Infix functions},language=Kotlin,label=code:infix]
infix fun Int.plus(that: Int): Int {
    return this + that
}
// Invocation:
val x = 5 plus 6  // Infix notation
val y = 5.plus(6) // This does the same as above
\end{lstlisting}

The second concept you must understand is treating objects as tokens. Objects can be utilized in various ways, but what's important for this section is their ability to create special tokens. You can define two functions with the same name but different arguments. If these arguments are of this kind of tokens, they can be used for pattern matching as shown in the \ref{code:patternobj} code snippet.

\begin{lstlisting}[caption={Pattern matching},language=Kotlin,label=code:patternobj]
object Token1
object Token2

fun foo(o: Token1) {
    ...
}
fun foo(o: Token2) {
    ...
}
// Invocation
val x = Token2
foo(x)      // Runs the second foo function
foo(Token1) // Runs the first foo function
\end{lstlisting}

Combining these Kotlin language features can be used to create fluent, natural language-like constructs. The \ref{code:jqx} code snippet demonstrates how these language concepts are employed to implement the JSON parsing aspects of the Konstrainer DSL.

\begin{lstlisting}[caption={jqx implementation},language=Kotlin,label=code:jqx]
infix fun JsonElement.jqx(selector: String): JsonElement {
    ...
}

object int
object bool
object double
object string

infix fun JsonElement.parseAs(type: int): Int? = 
    nullGuard { /* Function defining how to get JSON field as Int*/ }
infix fun JsonElement.parseAs(type: bool): Boolean? = nullGuard { ... }
infix fun JsonElement.parseAs(type: double): Double? = nullGuard { ... }
infix fun JsonElement.parseAs(type: string): String? = nullGuard { ... }

// Usage:
val x = jsonElement jqx "/foo/bar" parseAs string
\end{lstlisting}

`jqx' is an infix function with a return type of JsonElement. `parseAs' is also an infix function with the type JsonElement as a receiver, so `jqx' and `parseAs' commands can be chained. There is a token created for every the primitive types that can occur in a JSON, and there is a corresponding `parseAs' function for each token. This setup facilitates the extraction of all types from a JSON structure. The `nullGuard' is a private function that return null if the conversion can not happen, for instance, due to a type mismatch or non-existing JSON field. Otherwise, it returns the converted value. The conversion function is defined in its block.

\subsection{Unmarshallers}



\subsection{DslMarker annotation}


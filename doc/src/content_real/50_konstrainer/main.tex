\chapter{Konstrainer DSL}

The main top level keyword is the `server', which declares a new agent. It is called server, because a web server will be generated from it. It takes 1 argument, the name of the server, which must be in kebab-case\footnote{kebab-case: \url{https://developer.mozilla.org/en-US/docs/Glossary/Kebab_case}}, and opens a block. To be precise it takes 2 arguments, the second argument is a lambda, but in this context it is easier to refer to the last lambda argument of the server function as the `block of the server keyword', or the `server block'.

From now on, I will always use this terminology whenever referring to the last lambda argument of a function.

For the code to compile, we also need to import the DSL library, and define a package. The \ref{code:server_boilerplate} code snippet shows the basic template of an agent.

\begin{lstlisting}[caption={Template of a DSL file},language=Kotlin,label=code:server_boilerplate]
package me.btieger

import me.btieger.dsl.*

val serverName = server("server-name") {

}
\end{lstlisting}

Inside the server block we get access to three new keywords: `clusterRole', `report', `webhook'

The `clusterRole' is a String property. With it, we can assign an existing cluster role to the agent. The cluster role binding will be automatically generated by the Konstrainer-Core when the agent is deployed. If the agent sends any request to the Kubernetes API, it needs to have authorization to perform that action. With the Konstrainer helm chart a clusterrole is installed too, which gives reed access to all Kubernetes resources. We can reference this clusterrole with the `ReadAny' keyword.

\begin{lstlisting}[caption={Usage of the clusterRole keyword},language=Kotlin,label=code:clusterrole_usage]
val server1 = server("server1") {
    clusterRole = ReadAny
}
val server2 = server("server2") {
    clusterRole = "my-custom-cluster-role"
}
\end{lstlisting}


The `report' keyword opens a new block. In this block we can create a report from the state of cluster. We can only use the report keyword once in each server block.

The `webhook' keyword defines a webhook which listens to certain events in the cluster. This can be used to create warnings, modify or reject some actions.

\subfile{content_real/50_konstrainer/1_report.tex}

\subfile{content_real/50_konstrainer/2_webhook.tex}

\section{Summary and additional features}



\subfile{content_real/50_konstrainer/lang_tree.tex}